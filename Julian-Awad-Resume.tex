%----------------------------------------------------------------------------------------
%	PACKAGES AND OTHER DOCUMENT CONFIGURATIONS
%----------------------------------------------------------------------------------------

\documentclass{resume} % Use the custom resume.cls style

\usepackage{hyperref}
\usepackage[utf8x]{inputenc}
\usepackage{verbatim}

\hypersetup{
    colorlinks=true,
    linkcolor=black,
    filecolor=black,
    urlcolor=black,
    pdfpagemode=FullScreen,
    }
\urlstyle{same}

\usepackage[left=0.3in,top=0.4in,right=0.3in,bottom=0.4in]{geometry} % Document margins
\newcommand{\tab}[1]{\hspace{0.3\textwidth}\rlap{#1}}
\newcommand{\itab}[1]{\hspace{1em}\rlap{#1}}
\name{\bf Julian Awad} % Your name

%\address{Clearances Held (Canada): NATO Secret \\ Controlled Goods Program \\ Enhanced Reliability}
\address{(613)-806-2681 \\ julian.awad@queensu.ca \\ www.linkedin.com/in/julian-awad}
\begin{document}

%----------------------------------------------------------------------------------------
%	WORK EXPERIENCE
%----------------------------------------------------------------------------------------

\begin{rSection}{Experience}

\begin{rSubsection}{Propulsion and Payload Engineer}{September 2021 - Present}{Queen's Rocket Engineering Team}{}
\item Designing the team's first 3D-printed Hybrid Rocket Engine to compete in the SA Cup 10,000ft SRAD category
\item Created Standard Operating Procedures (SOP) documentation for the safe operation of a cold-flow and hot-fire test, complete with hazard assessment, risk mitigation, and contingency planning
\item Designing an autonomous glider payload to be launched from the rocket at apogee and collect atmospheric data, controlled by ArduPilot

\end{rSubsection}

\begin{rSubsection}{Lockheed Martin - Hardware Engineering Intern}{May 2021 - August 2021}{Rotary Missions Systems - CSC Project}{NATO Secret, Controlled Goods Program}

\item Performed detailed SolidWorks FEA analysis to validate equipment to Military Standard 901D
\item Created an Excel VBA tool to generate shock response spectra from an impulse function for shock \& vibe testing, resulting in a user-friendly program regularly used across the CSC Project
\item Accomplished overall 2x cost reduction and 4x time savings by performing detailed make-vs-buy analysis on electronics enclosures and presenting to senior engineers
\item Created and presented a whitepaper detailing the thermal, ingress protection, shock resistance, safety, and maintenance considerations of mounting locations for electronics enclosures
\item Clearances Held: NATO Secret, Controlled Goods Program, Enhanced Reliability

\end{rSubsection}

\begin{comment}

\begin{rSubsection}{Department of National Defense - Engineering Intern}{May 2020 - September 2020}{}{}

\item Documented and presented key specifications on armored patrol vehicles for 411 vehicles in 69 variants
\item Reworked procurement documents based on technical requirements from multiple military bases
\item Proofread english-to-french translations of contracts to ensure correctness

\end{rSubsection}
\end{comment}
\end{rSection}

\begin{rSection}{Projects \& Publications}

\begin{rSubsection}{Undergraduate Publication}{September 2021 - December 2021}{An Investigation of Magnetic Radiation Shields for Human Space Habitats}{Awad et al.}
\item Designed and conducted an experiment over 6 weeks to measure the viability of a superconducting magnet as an active radiation shield for lightweight space travel applications
\item Manufactured a vacuum chamber with a cooling tube configuration, wire feed-through, and a beta particle detector capable of maintaining a vacuum of 0.1 Pa to minimize particle stopping power and reduce condensation
\item Designed superconducting magnet configurations made of superconducting YBCO tape with a vacuum-tight cooling system to maintain critical temperatures of 77K
\item Created a Python program to perform in-depth analysis of the raw data, including noise filtering, curve fitting, and extrapolation to demonstrate clear trends

\end{rSubsection}

\begin{comment}
\begin{rSubsection}{Co-Founder, PolyTwist Designs}{November 2015 - Present}{\url{www.polytwist.xyz}}{}
\item Co-founded a small business designing and manufacturing original Rubik's-Cube-style puzzles with unique mechanisms, challenges, and solutions using FDM 3D Printing and SolidWorks
\item Designed and manufactured several novel products end-to-end resulting in 16+ original designs
\item Created and maintained a website and online shop resulting in \$20,000 in sales of 16+ products over three years
\item Negotiated a partnership with Rubik's Brand Ltd. to mass-produce a product, involving the design stages to manufacturing through injection molding and packaging design

\end{rSubsection}
\end{comment}

\end{rSection}

%----------------------------------------------------------------------------------------
%	TECHNICAL SKILLS SECTION
%----------------------------------------------------------------------------------------

\begin{rSection}{Technical Skills}
\begin{tabular}{ @{} >{\bfseries}l @{\hspace{6ex}} l }

Mechanical Design & SolidWorks + Simulation, FDM 3D Printing, Autodesk Inventor\\
Programming & Python, SciPy, MATLAB/Simulink, LabVIEW, Git, \LaTeX, C\\
Languages &  English, French (Native Bilingual), Spanish (Working Proficiency)

\end{tabular}
\end{rSection}

%----------------------------------------------------------------------------------------
%	EDUCATION SECTION
%----------------------------------------------------------------------------------------
\begin{rSection}{Education}

\begin{rSubsection}{Faculty of Engineering, Queen's University, Kingston ON}{May 2023 (Expected)}{}{}{}
  \item Candidate for Bachelor of Engineering Physics, Mechanical Stream
  \item Dean's List with Honours - GPA of 3.75/4.3
\end{rSubsection}
\begin{rSubsection}{Publications}{}{}{}
  \item Julian Awad, Nikhil Menda, William Conway, and David Puddy, “Investigation of Magnetic Radiation Shields for Human Space Habitats,” J. Undergrad. Eng. Phys. Phys. Exp. Queens, Section 1, Vol 3.

\end{rSubsection}

\end{rSection}


\end{document}
